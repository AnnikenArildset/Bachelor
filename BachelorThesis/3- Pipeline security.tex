\newpage
\thispagestyle{empty}
\mbox{}
\chapter{Pipeline security}
\label{Pipeline security}
\section{Introduction}
Pipeline security is a large part of securing software development, from source code to production deployment. Pipeline security is categorized into two aspects - security in the pipeline and security of the pipeline \cite{inofpipeline}. In many ways implementing both security in the pipeline and of the pipeline will increase the overall security of the pipeline. Further, the group will look closer into essential security measures within these two. 
\\~\\
Figure \ref{fig: Pipeline without any security measures} displays a \gls{pipeline} without any security measures implemented. The initial four stages in the pipeline are considered the implementation phase in the \acrshort{sdlc}. Furthermore, the security testing stage in the pipeline is considered the testing phase, the deployment stage is the deployment phase, and so on. The following chapter will present the different measures that will be added to the pipeline. In the end, a complete and secure pipeline will be achieved.

\vspace{2mm}
\begin{figure}[H]
    \centering
    \includegraphics[width=0.8\columnwidth]{Images/SecurePipeline-Page-3.drawio.png}
    \caption{Pipeline without any security measures}
    \label{fig: Pipeline without any security measures}
\end{figure}

\section{Security in the pipeline}
Security in the pipeline is the process of implementing different measures and controls to protect the code that is being sent through the pipeline from various security threats. The pipeline commonly consists of multiple phases like code development, testing, and deployment, all of which may be susceptible to various security threats, like unauthorized access, data breaches, malware, and denial-of-service attacks. Security in the pipeline is crucial to secure the integrity and confidentiality of software applications and data.
Below are tools that can be used for security in the pipeline. For a more detailed explanation of the different tools, see chapter \ref{chap:Tools}. 

\subsection{Code Scanning}
\label{Code Scanning}
According to Microsoft's best practices for secure \acrshort{sdlc} \cite{microsoftSDLCpractices}, a SAST tool should be included in the pipeline. A SAST tool can be for example code scanning. Code scanning is a security measure where code is analyzed with the help of a tool to find security vulnerabilities and coding errors. Code scanning serves as a preventive measure against developers introducing new issues. In Figure \ref{fig: Pipeline with implemented SAST scan}, the \acrshort{sast} scan is set to be in GitHub. By doing this scan there, it is done at the earliest stage possible, before the code is moved forward in the pipeline, eliminating vulnerabilities as soon as possible.

 \vspace{2mm}
\begin{figure}[H]
    \centering
    \includegraphics[width=0.8\columnwidth]{Images/pipeline2.png}
    \caption{Pipeline with implemented SAST scan}
    \label{fig: Pipeline with implemented SAST scan}
\end{figure}

\subsection{Scan Dependencies and Open Source Libraries}
\label{Scan Dependencies and Open Source Libraries}
Dependencies can be divided into two parts: \textit{direct} and \textit{transitive}. A direct dependency is directly referenced software component in an application. A transitive one is a functional software component necessary for an application's direct \gls{dependency}. These \gls{dependency} may have their own set of direct and indirect \gls{dependency}, resulting in a recursive tree of transitive \gls{dependency} affecting the application. This essentially means that the \gls{dependency} used in the code may be linked to numerous additional dependents, creating a large supply chain. To secure these supply chains, in addition to vulnerability scanning, the company can create a clear policy for evaluating and managing \gls{dependency}, including criteria for selecting secure and trustworthy libraries and frameworks. They should also limit the use of unnecessary or outdated \gls{dependency}, as these can increase the attack surface and create unnecessary risk. \cite{googledependency}
\\~\\
All dependencies, open-source libraries, and third-party \gls{artifact}s that have been utilized should be validated. To validate a file's integrity, compare the signature of an artifact to the signature generated by the artifact provider. This comparison helps in detecting any unauthorized alterations, tampering, or corruption of dependencies that may occur as a result of a \gls{maninthemiddle} attack or a compromise of the artifact repository. If any third-party software was implemented in the application it's important to conduct an \acrshort{sca} scan using suitable tools to identify whether any vulnerable open-source software was used. \cite{bestpracticeSupplyChain}
\\~\\
The \acrshort{sca} is placed in combination with the \acrshort{sast} tool in Figure \ref{fig: Pipeline with implemented SCA scan}. Similarly to \acrshort{sast}, it is done in GitHub to patch the vulnerable \gls{dependency} early in the implementation phase.

\vspace{2mm}
\begin{figure}[H]
    \centering
    \includegraphics[width=0.8\columnwidth]{Images/pipeline3.png}
    \caption{Pipeline with implemented SCA scan}
    \label{fig: Pipeline with implemented SCA scan}
\end{figure}

\subsection{Secret Scanning}
To prevent or identify accidental exposure of \say{secrets}, like access tokens, SSH keys, or other credentials, secret scanning should be executed on the repository were the source code is stored. Such secrets can give unwanted access to accounts, software, or cloud providers, among other things. With access to for example cloud credentials, a threat actor could, among other harmful attacks, scale up the use of various costly resources, costing a company much more than what they have budgeted for \cite{GitGuardianexploitexample}. According to GitGuardian's early report on secret leaks \cite{GitGuardiansecretsprawl}, they detected 10 million secrets leaked in 2022, which is an increase of 67\% from 2021. The same report showed 1 in 10 GitHub authors had exposed a secret in their repository. This shows an increasing issue with secret leaks. Secret scanning tools can be used to find these vulnerabilities, and alert developers to potential security risks. \cite{GithubSecretScanning} 
\\~\\
In Figure \ref{fig: Pipeline with implemented secret scan}, secret scanning is recommended to be done together with \acrshort{sast} and \acrshort{sca} for the same reason, explained in sections \ref{Code Scanning} and \ref{Scan Dependencies and Open Source Libraries}.

\vspace{2mm}
\begin{figure}[H]
    \centering
    \includegraphics[width=0.8\columnwidth]{Images/pipeline4.png}
    \caption{Pipeline with implemented secret scan}
    \label{fig: Pipeline with implemented secret scan}
\end{figure}

\subsection{Dynamic scanning}
In software best practices, it is recommended to run multiple tests and scans to identify bugs and errors - where one of these tests is \acrlong{dast} (\acrshort{dast}) \cite{bestpracticeSupplyChain}. This scanning method tries to penetrate the application, attempting to identify vulnerabilities and weaknesses in it. A specialized tool for \acrshort{dast} scan, such as OWASP ZAP, can be implemented to identify security risks like
\gls{Cross-site scripting}, \gls{SQL-injection} or path traversal.\cite{dynamictesting}


\subsection{Manual security testing}

%alt kan automatiseres men kan være lurt å ha manuel testing og

Even though \acrshort{dast} can be used to identify potential vulnerabilities, certain types of threats may go undetected. For this reason, the company should engage a red team, which is a group of experts capable of performing penetration testing. A penetration test will provide a more realistic test, as it simulates a real-world attack, detects more complex vulnerabilities, and provide a more comprehensive view of an application's security posture. A penetration test can also function as a validation of the \acrshort{dast} scan, as it can help determine if the vulnerability can be exploited and the potential impact of the vulnerability. \cite{dastpentesting}
\\~\\
In Figure \ref{fig: Pipeline with implemented DAST scan and pentesting} the \acrshort{dast} scan and penetration test is placed later in the pipeline because these tests are executed on a running application. Therefore, the tests can not be done any earlier, yet they should be done before the application is deployed.
\vspace{2mm}
\begin{figure}[H]
    \centering
    \includegraphics[width=0.8\columnwidth]{Images/pipeline5.png}
    \caption{Pipeline with implemented DAST scan and pentesting}
    \label{fig: Pipeline with implemented DAST scan and pentesting}
\end{figure}


\section{Security of the pipeline}
Security of the pipeline in the short term is the security measures that are taken into consideration when securing the pipeline itself. This includes not only securing the pipeline itself but also the infrastructure, components, and network that are used to process the code that goes through the pipeline. Securing the pipeline ensures that the code is not tampered with during the process of going through the pipeline. 


\subsection{Branch Protection}
\label{branchprotection}
Branch protection is a feature of GitHub, that enforces different rules and requirements for specific branches in the repository. The purpose of branch protection is to maintain the stability and security of the code, this is done by ensuring that all changes done to the branch have gone through the proper steps before being merged into the main codebase. Below are the different branch protection features that can be enabled in GitHub. \cite{ProtectedBranches}
\\
\subsubsection{Require a pull request before merging}
Administrators of the repository can add rules to the repository which restrict pull requests to have a specific number of people approving the changes before merging to a protected branch. Administrators can allow users with written permissions to do the approving as well as users considered to be code owners. 

It is under this type of protection that the "four eyes" principle is applied. Since this type of protection require that at least two people approve the merge, this includes the person itself doing the changes, this principle is a controlling mechanism that increases the security measures. 
\\
\subsubsection{Require status checks to pass before merging}
When users work together in the same repository, it is important to maintain the high quality of the code that's being pushed and merged. Enabling "require status checks to pass before merging" allow the administrators of the repository to set certain criteria that need to be fulfilled before the code is merged. 

\subsubsection{Require conversation resolution before merging}
When working together on the same repository it is important to have clear communication and collaboration  A way to secure this is to enable "require conversation resolution before merging". This allows all discussions regarding for example issues or pull requests that need to be properly resolved before any merging happens. 
\\
\subsubsection{Require signed commits}
Enabling require signed commits can be considered a security measure that secures that changes in the code have not been tampered with. 
To be able to have secured signed commits, all commits pushed to the repository must be signed with a \acrlong{gpg} key. A \acrshort{gpg} key is a unique digital signature that in short terms verifies the person that is committing to the repository is who they say they are. 


\subsubsection{Require linear history}
When enabling this security feature, it ensures that all changes done in the repository are done in a specific order, one after another without creating any branches or forks. This will as a result make it easier to keep track of all the changes that have been done and minimizes the chances of mistakes. Enforcing such measures restricts users from merging their changes with the main branch. 

\subsubsection{Require deployments to succeed before merging}
Require deployments to succeed before merging is another feature in GitHub, that allows the users to enforce that different required checks that have been created, such as pre-merge checks or automated tests, are passed before a pull request can be merged into the main branch.


\subsubsection{Lock branch}
Lock branch as the name implies allows the users to lock a branch in a repository, which will prevent changes from being made to the branch. This can be useful if there are situations where the branch needs to be protected from unauthorized changes or to be deleted. 


\subsubsection{Do not allow bypassing the above settings}
This feature, in short, prevents users in a repository from bypassing required checks that have been created or other restrictions. For example, if a user with admin access has configured a branch protection rule which requires that all pull requests reviews and status checks needs to be passed before merging, enabling this feature will then tell the user to go through these reviews and checks before any changes are done directly to the branch. 

\vspace{2mm}
\begin{figure}[H]
    \centering
    \includegraphics[width=0.8\columnwidth]{Images/pipeline6.png}
    \caption{Pipeline with implemented branch protection rules}
    \label{fig: Pipeline with implemented branch protection rules}
\end{figure}

\subsection{Access Control}
Access control involves implementing measures to regulate the individuals who are allowed to access certain resources like GitHub, as well as determining the appropriate level of access for each individual. When applying these measures, one should follow the "least privilege" principle. According to NIST, this is \textit{"the principle that a security architecture should be designed so that each entity is granted the minimum system resources and authorizations that the entity needs to perform its function"}\cite{leastprivilege}. In Github, access is controlled by permission - which is the capability to execute a particular task. Employees can also be assigned to roles. A role is a type of permission that you can grant to an employee or a team. \cite{accesscontroll}
\\~\\
A security measure that works like access control is \acrlong{mfa} (\acrshort{mfa}). It involves requiring users to provide additional information during the login process, not just a password. In addition to for example a password, the user must use an additional authentication factor, such as a hardware token, a code sent to their phone, or an app installed on their phone.
This procedure can help to avoid unauthorized access to a company's systems. To ensure overall system security, it is critical to enable this measure across all system components, as securing only one part of the system will not be effective if another part remains unprotected. \cite{MFA}

\vspace{2mm}
\begin{figure}[H]
    \centering
    \includegraphics[width=0.8\columnwidth]{Images/pipeline7.png}
    \caption{Pipeline with implemented access control}
    \label{fig: Pipeline with implemented access control}
\end{figure}
 
\subsection{File Storage and Preservation}
To be able to maintain proper security of the pipeline, file storage and preservation can be considered a large part of this. File storage and preservation help secure the pipeline over time, by doing regular backups of the data related to the pipeline. By maintaining redundant copies of critical data, the risk of data loss due to attacks or other security breaches can be minimized. 

Some other important aspect of file storage and preservation is access control, which should be restricted so one only has the necessary access. As a result, this can minimize the chances of data tampering or data theft.

\section{Security in maintenance}
Once the deployment is complete, the application is then transferred to the cloud environment in \acrlong{aws}. It is essential at this stage to keep the security up to date, to ensure that the application and data are protected. AWS offers a range of best practices organizations can follow to decrease risks associated with cloud computing and ensure that the  \acrshort{aws} environment is secure.

After the deployment is successful, it's essential to ensure that the infrastructure is secure and optimized for performance. Regular maintenance and updates are important to solve security issues and add new features to the system. It's also important to perform regular backups and disaster recovery testing to ensure that the data and applications are protected. 
Staying up-to-date with maintenance and testing can minimize downtime and improve system reliability. 

When the application is in the cloud environment, it is important to continuously monitor it. This is to identify any issues or potential threats. This includes monitoring for errors, performance issues, security vulnerabilities, and more. \acrshort{aws} offers specialized tools for this type of monitoring, like Amazon cloudwatch\footnote{Available at: https://aws.amazon.com/cloudwatch/} and AWS X-ray\footnote{Available at: https://aws.amazon.com/xray/}. 

Various security tests, such as security scans and penetration testing, should be performed during the development phases to identify potential vulnerabilities and address them before deployment. However, it is also essential to continue testing after deployment to ensure that any new vulnerabilities are identified and addressed as soon as possible. Regular security scans and penetration testing can significantly reduce the risk of exploitation. By conducting these tests regularly, the organization can keep track of any possible security problems and take proactive measures to mitigate them.

As a protective measure, the organization should set up a web application firewall like AWS WAF\footnote{Available at: https://aws.amazon.com/waf/}, to prevent malicious application attacks such as \gls{SQL-injection}, \gls{Cross-site scripting} and other attacks. AWS's WAF service allows the organization to customize rules and access control lists based on the company's needs and risk models. This makes it possible to provide web application security with more customization and specificity. Securing an application against \gls{ddos} is an important additional step. AWS Shield\footnote{Available at: https://aws.amazon.com/shield/} is a service that an organization can utilize to achieve this. \cite{awsafterdep}

Security is an important aspect of the SDLC's maintenance phase, which begins after the application has been deployed. It is critical to keep up with regular maintenance and testing during this phase, as well as to take proactive measures to mitigate any issues that may arise. Organizations can identify and address potential security vulnerabilities in this manner before they become major issues. This is especially important when deploying applications to the cloud because the security landscape can be complex and ever-changing. Best practices such as regular security audits, vulnerability scans, and the implementation of security patches can help to ensure that the application remains secure and protected against potential threats. Monitoring for unusual or suspicious activity can also aid in the detection and prevention of security breaches. Organizations can help to ensure the continued reliability and security of their AWS applications by prioritizing security throughout the maintenance phase.
\newpage
\section{Finished pipeline}
This is the finished pipeline after all security measures are included.

\vspace{2mm}
\begin{figure}[H]
    \centering
    \includegraphics[width=0.8\columnwidth]{Images/pipeline9.png}
    \caption{Pipeline with all security measures implemented}
    \label{fig: Pipeline with all security measures implemented}
\end{figure}




\newpage
\section{Frameworks}
\subsection{Introduction}
A framework is \textit{\say{a supporting structure around which something can be built}} \cite{FrameworkDefinition}, and can be used in connection with securing the \acrshort{sdlc}. Frameworks outlined in this chapter are not intended to be rigid regulations but rather valuable recommendations for software developers to enhance security measures. By implementing these suggestions, developers can confidently improve the security of their software.

\subsection{Supply-chain Levels for Software Artifacts}
\acrlong{slsa} (\acrshort{slsa})\footnote{Available at \url{https://slsa.dev/}} is a framework for securing the software supply chain created by Google in collaboration with OpenSSF\footnote{Available at \url{https://openssf.org/}} \cite{SLSAgeneral}. The framework is made into a common vocabulary checklist for developers to evaluate the security of the software they are creating. \acrshort{slsa} is organized into tracks and levels. The levels refer to the increasing security guarantee of the supply chain, the highest level being level 3. The levels are split further into tracks. Tracks are certain aspects of the supply chain, for example, the Build track, which currently is \acrshort{slsa}s only track. 

\vspace{2mm}
\begin{figure}[H]
    \centering
    \includegraphics[width=0.8\columnwidth]{Images/slsalevels.png}
    \caption{\acrshort{slsa} levels for the Build track}\cite{SLSAlevels}
    \label{fig: SLSA levels for the Build track}
\end{figure}
Currently, there are three levels split into one track. To achieve the different build levels, the developers have to do the following:
\\~\\
To achieve \acrshort{slsa} Build Level 1, the developers must use a consistent build process, which can quickly be adopted. Additionally, it is essential to generate \gls{provenance} automatically on the build platform, which describes how the artifact was built. This includes information on the entity responsible for building the package, the specific build process used, and the top-level inputs utilized during the build process.
\\~\\
In order to reach \acrshort{slsa} Level 2, all Level 1 requirements must be in place. Further, the build has to be run on a platform that signs the \gls{provenance}. Finally, this \gls{provenance}'s authenticity must also be verified.
\\~\\
Similarly to Level 2, all previous level requirements must be achieved to get to \acrshort{slsa} Level 3. In addition, the build platform has to have controls to secure the secrets used for signing \gls{provenance} and prevent runs from the same project from impacting each other. 

\subsection{Secure Software Development Framework}
\label{ssdf}
\acrlong{ssdf} (\acrshort{ssdf})\footnote{Available at \url{https://csrc.nist.gov/Projects/ssdf}} is a framework consisting of practices for a secure software development, created by \acrlong{nist} (\acrshort{nist}). The organization should integrate the \acrshort{ssdf} into their already existing software development practices. \acrshort{ssdf} does not specify how each practice should be implemented. It emphasizes the outcome of the practices rather than how to perform them. Organizations in any sector or community can use the \acrshort{ssdf}, regardless of their size or level of cybersecurity competence. This framework is intended to be user-friendly and adaptable, making it appropriate for a wide range of businesses with varied levels of cybersecurity knowledge. Organizations can use the \acrshort{ssdf} to adopt secure software development practices and reduce the risk of potential security vulnerabilities. The framework does not introduce new practices or define new terminology. However, it presents a set of high-level practices based on known standards, guidelines, and documents relevant to secure software development practices. 
\\~\\
The benefits of describing the practices at a high level include that they can be used by organizations in every industry and community, despite their size or level of cybersecurity knowledge. It can also help companies that buy and use software understand the secure software development methods used by their suppliers. All the practices are described in the framework\footnote{Latest version: \url{https://nvlpubs.nist.gov/nistpubs/SpecialPublications/NIST.SP.800-218.pdf}}.
\\~\\
There are four groups into which the practices are divided \cite{ssdf}:
\begin{itemize}
  \item \textbf{Prepare the Organization (PO)}: \textit{\say{Organizations should ensure that their people, processes, and technology are prepared to perform secure software development at the organization level. Many organizations will find some PO practices to also apply to subsets of their software development, like individual development groups or projects.}}
  \item \textbf{Protect the Software (PS)}: \textit{\say{Organizations should protect all components of their software from tampering and unauthorized access.}}
  \item \textbf{Produce Well-Secured Software (PW)}: \textit{\say{Organizations should produce well-secured software with minimal security vulnerabilities in its releases.}}
  \item \textbf{Respond to Vulnerabilities (RV)}: \textit{\say{Organizations should identify residual vulnerabilities in their software releases and respond appropriately to address those vulnerabilities and prevent similar ones from occurring in the future.}}
\end{itemize}

Each practice definition has the following components \cite{ssdf}:
\begin{itemize}
  \item \textbf{Practice}: Name of the practice with a unique identifier, with a description. 
  \item \textbf{Task}: One or more steps may be required to carry out a procedure.
  \item \textbf{Notional Implementation Examples}: A selection of tools, procedures, or approaches is presented that may aid in the execution of tasks. It should be noted that these examples are not exhaustive, and their use is not obligatory. In addition, some examples may not be relevant to specific companies or circumstances.
  \item \textbf{References}: References towards established secure development practice documentation and their mappings to specific tasks. Not all references will be relevant in all cases of software development. 
\end{itemize}

