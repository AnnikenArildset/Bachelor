\section{Theory}
\subsection{Introduction}
\subsection{\acrlong{sdlc}}

\subsection{SAST}
\acrlong{sast} is a type of software security testing tool that analysis source code of an application to identify different potentially security vulnerabilities within the code. This method of testing usually takes place during the development phase of the \acrlong{sdlc}. The primary purpose of this method is to identify and remediate security issues before the application actually is deployed. \cite{sast}

\acrshort{sast} tools scan source code for known security threats, such as \acrlong{xss}, SQL injections and buffer overflows.\acrshort{sast} tools also give warnings on any security weaknesses that may lie in the code that potentially can be exploited. After the tool has gone through the code it generates a report that contains the different vulnerabilities that it has identified, including a more in-depth description of the vulnerability and a remediation on how to fix it. 

One of the advantages with \acrshort{sast} is that it gives detailed information about the source of the vulnerability, which gives the developer a better understanding on have to fix the issue. 

There are of course some limitations to \acrshort{sast} tools as well. For example, it can only detect vulnerabilities that are present in the source code. This means that it cannot detect vulnerabilities that that results from the interaction between different components of an application.


\subsection{DAST}
Compared to \acrlong{sast}, \acrlong{dast} is also a type of software security testing tool. However, what \acrshort{dast} does is that it evaluates the security of an application by performing security assessments of a running instance of the application. Unlike \acrshort{sast} which analyzes the source code of an application, \acrshort{dast} evaluates the application as its being used, this include the interaction of different components and the runtime environment. 

\acrshort{dast} simulates real-world attacks, which it does by sending malicious requests and inputs to the application it is testing and then monitoring the responses. In the end, the tool generates a report that includes the different vulnerabilities that was identified, including a more in-depth decription as well as a remedation on how to fix the issue. \cite{dast}

An advantage with \acrshort{dast} is that it can identify security issues that is not detectable through \acrshort{sast}, this can for example be interactions between different components. Another advantage is that with \acrshort{dast}, it can identify different vulnerabilities that gets triggered, for example when the application is under heavy load or when are specific inputs received.

However, there are of course some limitations with \acrshort{dast} as well, one being that it can only detect vulnerabilities that are present in the deployed version of the application and cannot give in-depth description on vulnerabilities that lie in the source code.

\subsection{SCA}
\acrlong{sca} is compared to the two others also a type of software security testing tool. What \acrshort{sca} does is that it analyzes the dependencies of a software application to identify and manage potential security risks. The main objective of the \acrshort{sca} is to identify third-party components that may contain security vulnerabilities. \cite{sca}

What \acrshort{sca} does is that it scan the application's code to identify all of its dependencies, including the different versions of the components used. It then cross-references these dependencies to different databases that include known vulnerabilities. It then generate a report containing any potential risk. In comparison to the others, the report also includes an in-depth description of the vulnerability as well as a recommendation to update the components to newer versions or replacing these. 

An advantage with \acrshort{sca} is that it can quickly identify risks that may be introduced from third-party components. It is rather common that modern applications relies on a large number of  different dependencies, which therefore make \acrshort{sca} pretty useful. It can provide a comprehensive view of the security risks associated with an application and help developers make informed decisions about the security and their applications. 

However, it is important to remember that \acrshort{sca} does not always have access to updated information, and may give some false-positives about vulnerabilities that doesn't necessary exist anymore. 

\subsection{Software Security Testing Approaches}
\subsubsection{Black Box Testing}
Black box testing mainly focuses on functionality and behaviour of the application without knowing the structure and processes within. One can imagine the application being a black box, not being able to see what's inside, but only focusing on the the resulting output from your input. A software will pass the black box test if input gives the expected output. \cite{blackbox}

\subsubsection{White Box Testing}
White box testing focuses on the application from within. In such tests, the source code and infrastructure will be looked at. The testing consist of covering paths, statements and branches, among other things. 

