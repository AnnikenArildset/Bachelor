\section{Security in the pipeline}
Security in the pipeline is the process of implementing different measures and controls to protect the code that is being sent through the pipeline from various security threats. The pipeline commonly consists of multiple phases like code development, testing, and deployment, all of which may be susceptible to various security threats, like unauthorized access, data breaches, malware, and denial-of-service attacks. Security in the pipeline is crucial to secure the integrity and confidentiality of software applications and data.
Below are tools that can be used to in security in the pipeline and for a more in-depth explanation of the different tools, see chapter \ref{chap:Tools}. 

\subsection{Code Scanning}
Code scanning is a security measure where code is analyzed with the help of a tool to find security vulnerabilities and coding errors. Code scanning serves as a preventive measure against developers introducing new issues. During this step, you can perform a SAST scan using specialized tools that are designed to scan through code. 
\\
\\
GitHub provides an integrated code scanner called CodeQL. CodeQL extracts all source code into a relational database optimized for CodeQL analysis.  It can then be queried to identify insecure patterns in the code and other vulnerabilities. The user can take advantage of a large number of queries already made by other developers, or they can make their own. An example of a simple query could be getting the location of all method calls in the code.  For more information regarding code scanning see \ref{tab: Code Scanning}
 \cite{codeql}

 \vspace{2mm}
\begin{figure}[H]
    \centering
    \includegraphics[width=0.8\columnwidth]{Images/pipeline2.png}
    \caption{Pipeline with implemented SAST scan}
    \label{fig: Pipeline with implemented SAST scan}
\end{figure}

\subsection{Scan Dependencies and Open Source Libraries}

\gls{dependency} can be divided into two parts: direct and transitive. A direct dependency is directly referenced software component in an application. A transitive one is a functional software component necessary for an application's direct \gls{dependency}. These \gls{dependency} may have their own set of direct and indirect \gls{dependency}, resulting in a recursive tree of transitive \gls{dependency}affecting the application. This essentially means that the \gls{dependency} used in the code may be linked to numerous additional dependents, creating a large supply chain. To secure these supply chains, in addition to vulnerability scanning, the company can create a clear policy for evaluating and managing \gls{dependency}, including criteria for selecting secure and trustworthy libraries and frameworks. They should also limit the use of unnecessary or outdated \gls{dependency}, as these can increase the attack surface and create unnecessary risk. \cite{googledependency}

All dependencies, open-source libraries, and third-party \gls{artifact}s that have been utilized should be validated. To validate a file's integrity, compare the hash of an artifact to the hash value generated by the artifact provider. This comparison helps in detecting any unauthorized alterations, tampering, or corruption of dependencies that may occur as a result of a man-in-the-middle attack or a compromise of the artifact repository. If any third-party software was implemented in the application it's important to conduct an \acrshort{sca} scan using suitable tools to identify whether any vulnerable open-source software was used. \cite{bestpracticeSupplyChain}


\vspace{2mm}
\begin{figure}[H]
    \centering
    \includegraphics[width=0.8\columnwidth]{Images/pipeline3.png}
    \caption{Pipeline with implemented SCA scan}
    \label{fig: Pipeline with implemented SCA scan}
\end{figure}

\subsection{Secret Scanning}
To prevent or identify accidental exposure of "secrets", like access tokens, SSH keys, or other credentials, one should execute a secret scanning on a Git repository. Secret scanning tools, such as GitHub's secret scanning, can be used to find these vulnerabilities, and alert developers to potential security risks. \cite{GithubSecretScanning}

GitHub has a Secret Scanning Partner Program\footnote{https://docs.github.com/en/code-security/secret-scanning/secret-scanning-partner-program} that allows service providers to collaborate with GitHub to secure their secret token formats. This program uses secret scanning technology, which examines code repositories hosted on GitHub for any unintentional commits of the service provider's secret format. If a potential secret is discovered, it can be routed to the service provider's verify endpoint for additional investigation and management. The benefit of this program is the ability for organizations and products to give their users a better and more complete security solution for their GitHub code repositories. \cite{partnerprogram}


\vspace{2mm}
\begin{figure}[H]
    \centering
    \includegraphics[width=0.8\columnwidth]{Images/pipeline4.png}
    \caption{Pipeline with implemented secret scan}
    \label{fig: Pipeline with implemented secret scan}
\end{figure}

\subsection{Dynamic scanning}
In software best practices, it is recommended to run multiple tests and scans to identify bugs and errors - where one of these tests is \acrlong{dast} (\acrshort{dast}).\cite{bestpracticeSupplyChain} This scanning method tries to penetrate the application, attempting to identify vulnerabilities and weaknesses in it. One can implement a tool specialized for DAST scans, such as OWASP Zap, which can identify security risks like \gls{Cross-site scripting}, \gls{SQL-injection} or path traversal.\cite{dynamictesting}


\subsubsection{The Limitations of DAST Tools}
Even though \acrshort{dast} can be used to identify potential vulnerabilities, certain types of threats may go undetected. For this reason, the company should engage a red team, which is a group of experts capable of performing penetration testing. A penetration test will provide a more realistic test, as it simulates a real-world attack, detects more complex vulnerabilities, and provide a more comprehensive view of an application's security posture. A pen test can also function as a validation of the \acrshort{dast} scan, as it can help determine if the vulnerability can be exploited and the potential impact of the vulnerability. \cite{dastpentesting}

\vspace{2mm}
\begin{figure}[H]
    \centering
    \includegraphics[width=0.8\columnwidth]{Images/pipeline5.png}
    \caption{Pipeline with implemented DAST scan and pentesting}
    \label{fig: Pipeline with implemented DAST scan and pentesting}
\end{figure}

\subsection{Artifacts}



