\makeglossaries


\newacronym{nor}{NOR}{Norway}
\newacronym{ntnu}{NTNU}{Norwegian University of Science and Technology}
\newacronym{nbim}{NBIM}{Norwegian Bank Investment Management}
\newacronym{sdlc}{SDLC}{Software Development Life Cycle}
\newacronym{aws}{AWS}{Amazon Web Services}
\newacronym{sast}{SAST}{Static Application Security Testing}
\newacronym{dast}{DAST}{Dynamic Application Security Testing}
\newacronym{sca}{SCA}{Software Composition Analysis}
\newacronym{xss}{XSS}{Cross-site Scripting}
\newacronym{paas}{PaaS}{Platform as a Service}
\newacronym{saas}{SaaS}{Software as a Service}
\newacronym{iaas}{IaaS}{Infrastucture as a Service}
\newacronym{cve}{CVE}{Common Vulnerabiliteis and Exposures}
\newacronym{nvd}{NVD}{National Vulnerability Database}
\newacronym{cvss}{CVSS}{Common Vulnerability Scoring System}
\newacronym{cwe}{CWE}{Common Weakness Enumeration}
\newacronym{ide}{IDE}{Integrated Development Environment}
\newacronym{gpg}{GPG}{GNU Privacy Guard}
\newacronym{idea}{IDEA}{International Data Encryption Algorithm}
\newacronym{mfa}{MFA}{Multi-factor authentication}
\newacronym{iam}{IAM}{Identity access management}
\newacronym{cicd}{CI/CD}{Continuous integration and continuous deployment}
\newacronym{ec2}{Amazon EC2}{Amazon Elastic Compute Cloud}
\newacronym{amis}{AMIs}{Amazon Machine Images}
\newacronym{slsa}{SLSA}{Supply-chain Levels for Software Artifacts}
\newacronym{ssdf}{SSDF}{Secure software development framework}
\newacronym{nist}{NIST}{National Institute of Standards and Technology}
\newacronym{vpc}{VPC}{Virtual Private Cloud}
\newacronym{sns}{Amazon SNS}{Amazon Simple Notification Service}
\newacronym{cli}{CLI}{Command-Line Interface}
\newacronym{owasp}{OWASP}{Open Worldwide Application Security Project}
\newacronym{zap}{ZAP}{Zed Attack Proxy}
\newacronym{ssh}{SSH}{Secure Shell}
\newacronym{iac}{IAC}{infrastructure as code}

\newglossaryentry{buildspec}
{
    name=Buildspec,
    text= buildspec,
    description={Buildsec is a collection of build commands and related settings in a YAML format that CodeBuild uses to run a build}
}


\newglossaryentry{GUI}
{
    name=GUI,
    description={Stands for: Graphical User Interface and it is
a graphics-based operating system interface that uses icons, menus and a mouse (to click on the icon or pull down the menus) to manage interaction with the system}
}




\newglossaryentry{pipeline}
{
    name=Pipeline,
    text = pipeline,
    description={A pipeline is a process that drives software development through a path of building, testing, and deploying code, also known as \acrshort{cicd}}
}


\newglossaryentry{front-end}
{
    name=Front-end,
    description={The frontend is everything a user sees and interacts with when they click on a link or type in a web address. The web address is also known as at URL, or Uniform Resource Locator, and it tells what webpage should load and appear in the browser
    }
}


\newglossaryentry{Cross-site scripting}
{
    name=Cross-site scripting,
    text=cross-site scripting, 
    description={ Cross-site scripting attacks are a type of injection, in which malicious scripts are injected into otherwise benign and trusted websites
    }
}


\newglossaryentry{SQL-injection}
{
    name=SQL-injection,
    description={ SQL injections is a web security vulnerability that allows an attacker to interfere with the queries that an application makes to its database
    }
}


\newglossaryentry{Buffer-Overflow}{
    name = Buffer-overflow,
    text = buffer-overflow,
    description={Buffer overflow occurs when the volume of data exceeds the storage capacity of the memory buffer}
}


\newglossaryentry{dependency}{
    name = Dependencies,
    text = dependencies,
    description={A dependency refers to any external software component that an application relies on in order to function}
}

\newglossaryentry{artifact}{
    name = Artifact,
    text= artifact,
    description = {An artifact is any software package that is used as a building block for applications}
}


\newglossaryentry{ddos}{
    name = DDoS attack,
    %text = ddos attack,
    description = {A DDoS (Distributed Denial of Service) attack is an attack where multiple systems are used to flood a targeted website or online service with traffic, overwhelming its capacity to handle requests and making it unavailable to legitimate users}
}

\newglossaryentry{compute platform}{
    name = Compute platform,
    text = compute platform,
    description = {The environment where software is executed. This could be either the operating system or physical hardware}
}

\newglossaryentry{infrastructure as code}{
    name = Infrastructure as Code,
    text = infrastructure as code,
    description = {Infrastructure as Code (IaC) is the process of automating the supply of IT infrastructure through the use of a high-level coding language}
}

\newglossaryentry{provenance}{
    name = Provenance,
    text = provenance,
    description = {Set of metadata providing information about how the outputs were generated, which includes identifying the platform used and any external parameters involved in the process}
}

\newglossaryentry{shift-left}{
    name = Shift-left,
    text = shift-left,
    description = {Shift left is the practice of moving testing, quality, and performance evaluation early in the development process, often before any code is written}
}

\newglossaryentry{appspec}{
    name = Appspec,
    text = appspec,
    description = {An Application specification file, or appspec file, is a YAML or JSON formatted file containing instructions for how to deploy and configure the application}
}

\newglossaryentry{maninthemiddle}{
    name = Man-in-the-middle attack,
    text = man-in-the-middle attack,
    description = {A man-in-the-middle (MiTM) attack is a type of cyber attack in which the attacker secretly intercepts and relays messages between two parties who believe they are communicating directly with each other}
}


\newglossaryentry{idempotent}{
    name = Idempotent,
    text = idempotent,
    description = {Idempotence, in programming and mathematics, is a property of some operations such that no matter how many times you execute them, you achieve the same result.}
}

