\chapter{Deployment}
\section{Introduction}
\section{Code used in the pipeline}
For testing the group decided on using OWASPs Juice Shop, which is a deliberately vulnerable web application that is designed to help developers and others to learn about web applications security concepts. The code is designed to simulate a real-world application by having common vulnerabilities within the code. The intention is to encourage users to find these vulnerabilities and to exploit them and increase the understanding of web application security. \cite{owaspJuiceShop}
The code in the OWASP Jucie Shop is open source code that lies on Github and is written in TypeScript, which uses a Node.js server and Angular for frontend. \cite{owaspJuiceShopCode}
The code contains of different vulnerabilities, including SQL injections, cross-site scripting and many others. 
Overall, OWASP Juice Shop encourages users to improve their skills and it allows for customization and adaption for specific needs from the users. 


\section{Security when coding}
Here we can write something about security while coding, maybe get CodeQL to work so we can test that idk???

\section{Security when pushing to GitHub}

\subsection{Access control}
Something about access control

\section{Security when in Git}

\subsection{Configuring CodeQL}
When configuring CodeQL in a GitHib repository, one can either use a default setup or configure it themselves. 

