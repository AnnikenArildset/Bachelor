\newpage
\chapter*{Sammendrag}

\begin{tabbing}
\hspace{2cm}\=\hspace{3cm}\=\kill % set the tab stops

Tittel: \> \> Securing the Software Development Life Cycle \\
Dato: \> \> 22.05.2023 \\ 
\\
Deltakere: \> \> Anniken Arildset \\ \> \> Celina Brynildsen \\ \> \> Sebastian Hestsveen \\ \> \> Thea Urne \\
\\
Veileder: \> \> Filip Holik, Forsker/Vitenskapelig assistent, \\\> \> Institutt for informasjonssikkerhet og kommunikasjonsteknologi \\
\\
Oppdragsgiver: \> \>  Astri Marie Ravnaas, Norges Bank Investment Management (NBIM) \\
\\
Nøkkelord: \> \> Informasjonssikkerhet, SDLC, DevSecOps, AWS, GitHub \\
\\
Antall sider: \> \> Xx \\
Antall vedlegg: \> \> Xx \\
Tilgjenlighet: \> \> Åpen \\
\\
Sammendrag: \\\> \>Mange sikkerhetstiltak kan integreres i prosessen så snart kode lastes opp\\\> \> til GitHub, og det er flere scanninger som kan være utført under overgangen\\\> \> fra GitHub til skymiljøet for å sikre at sikkerheten er maksimert før man\\\> \> distribuerer applikasjonen til skyen. Gruppen har bygget en automatisert\\\> \> pipeline og lagt til verktøy og tiltak som anses som beste praksis i bransjen.\\\> \> Verktøyene ble valgt basert på brukervennlighet og tidligere dokumentasjon\\\> \> av lignende testing.

\end{tabbing}


