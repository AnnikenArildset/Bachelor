\chapter*{Sammendrag}
{\setstretch{1.30} % Adjust line spacing
\begin{tabbing}
\hspace{2cm}\=\hspace{1.5cm}\=\kill % set the tab stops

Tittel: \> \> Securing the Software Development Life Cycle \\
Dato: \> \> 22.05.2023 \\ 
\\
Deltakere: \> \> Anniken Arildset \\ \> \> Celina Brynildsen \\ \> \> Sebastian Hestsveen \\ \> \> Thea Urne \\
\\
Veileder: \> \> Filip Holik, Forsker/Vitenskapelig assistent, \\\> \> Institutt for informasjonssikkerhet og kommunikasjonsteknologi \\
\\
Oppdragsgiver: \> \>  Astri Marie Ravnaas, Norges Bank Investment Management (NBIM) \\
\\
Nøkkelord: \> \> AWS, DevSecOps, GitHub, Informasjonssikkerhet, Infrastruktur som kode,\\\> \> SDLC \\
\\
Antall sider: \> \> 86 \\
Antall vedlegg: \> \> 8 \\
Tilgjenlighet: \> \> Åpen \\
\\
Sammendrag: \> \>Denne rapporten fokuserer på å adressere viktigheten av å sikre systemer \\\> \>og applikasjoner ved å fokusere på implementering av sikkerhetstesting og\\\> \> integrering av verktøy i en utviklingspipeline. Rapporten gir et proof of \\\> \>concept for å bygge en sikker pipeline, og legger vekt på beste praksis og \\\> \>sikkerhetsverktøy for å oppdage sårbarheter tidlig i Software Development\\\> \> Life Cycle (SDLC). Funnene og anbefalingene kan bidra til industriens\\\> \> forståelse av å sikre SDLC og redusere trusler.
\end{tabbing}
}



