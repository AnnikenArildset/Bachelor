\chapter{Conclusion}
\section{Introduction}
This chapter includes a discussion of the work done, the techniques employed to achieve the intended outcomes, and several constructive suggestions to enhance the quality of the thesis.

%section{Reflection}


\section{The Work Process}

\subsection{Meetings}
The team held meetings with their supervisor, Filip Holik, every week. Although the meetings were typically scheduled for Wednesdays, they had to be postponed or canceled on occasion due to various circumstances. Meeting Minutes with the supervisor can be found in Appendix \ref{supervisormeeting}.
\\~\\
During the early stages of the project, the team met with the stakeholder every other week. However, as the project progressed, both parties agreed that weekly meetings would be beneficial. In addition, as the demand for help grew, the group required more regular assistance. Meeting Minutes with the stakeholder can be found in Appendix \ref{nbimmeeting}.  

\subsection{Scrum}
Throughout the thesis, the group followed the Scrum framework of agile project management, which required breaking the project into sprints lasting two to four weeks. Daily stand-up meetings were held to keep everyone informed, during which members delivered progress reports on the thesis and discussed plans for the day, including each member's allocated chores. While the group aimed to meet daily, this was only sometimes possible because some members had work obligations besides their studies.
\\~\\
Despite this, the team had sprint planning and retrospective sessions every two weeks to review progress and establish goals for the next two weeks. During the retrospective meetings, the team engaged in self-reflection by asking questions such as "What should we keep doing?" "What should we do more of?" "What should we do less of?" and "What should we stop doing?" to identify improvement areas for the next sprint cycle. During the sprint planning meetings, on the other hand, the group evaluated completed tasks and discussed what needed to be done before the next sprint period.
\\~\\
Initially scheduled for Fridays, the group discovered that moving their meetings to Mondays was more productive. In addition, starting a new strategy at the beginning of the week has proven to be more successful than commencing a new plan at the end of the week.
\\~\\
Following the Scrum framework helped promote good communication and teamwork among group members. Furthermore, the Kanban board was integrated into the Scrum Framework and proved to be quite helpful in tracking tasks that were ongoing, completed, or yet to be started. When a group member completed a task, there were always tasks on the Kanban board that needed to be completed. 

\subsection{Coordinated schedule}
In order to better coordinate their busy schedules, team members with different commitments, such as work and student associations, decided to implement a scheduling method. Doing this allowed them to quickly determine each other's availability and schedule meetings more efficiently. To achieve this, the team utilized the calendar feature within their Teams channel to schedule their activities and workdays.

\subsection{Draft Submissions}
The team set specified deadlines for producing multiple drafts at the start of the project. This approach tried to maintain constant development while avoiding last-minute delays. For instance, the group set an April 1st target for their first draft, which they could meet. As a result, both the supervisor and the stakeholder received the first draft on time. After a few days, the team got feedback and proceeded with their work. 
\\
Additionally, the group established a deadline for the final draft. Setting the final draft deadline on the 1st of May, three weeks before the submission date of the 22nd of May, gave the stakeholders and supervisor sufficient time to review the thesis thoroughly. 
\\
It is essential to mention that the group followed the plan comprehensively, allowing them to send in multiple drafts for review and refinement before the final submission. 



\subsection{Gantt Chart}
Since the group decided to change the scope during the project period, the original Gantt chart could not be followed. 
\\~\\
The research on tools was initially considered time-consuming, but with the changes made, this activity became a minor part of the thesis than anticipated. As a result, it took less time than what was first estimated. Furthermore, the \say{Testing tools} activity was removed since the team wanted to focus less on testing tools and more on integrating testing of the various tools into the pipeline-building process. 
\\~\\
The process of learning \acrshort{aws} and configuring the pipeline with Terraform took more extended time than initially expected.
The documentation for \acrshort{aws} and Terraform is extensive, but navigating it can be overwhelming. With so many different services and features offered by \acrshort{aws}, the group had to determine the best fit for their needs. In addition, building the pipeline required a significant amount of functionality to be in place for additional features to work correctly. 
\\~\\
However, the group followed the planned deadlines for the different draft submissions. As a result, the first draft was delivered on the 3rd of April in week 13, and the final draft on the 5th of May in week 18. 

\vspace{2mm}
\begin{figure}[H]
    \centering
    \includegraphics[width=1\columnwidth]{Images/gantt2.jpg}
    \caption{Original Gantt Chart}
    \label{fig: Original Gantt Chart}
\end{figure}

\vspace{2mm}
\begin{figure}[H]
    \centering
    \includegraphics[width=1\columnwidth]{Images/finished-gantt.png}
    \caption{Updated Gantt Chart}
    \label{fig: Updated Gantt Chart}
\end{figure}

\newpage
\subsection{Distribution of Work}
The group decided to divide the work into two parts at the start of the thesis to ensure that the thesis progressed continuously. The first part is practical, and the second is report writing. The responsibilities were assigned based on each group member's strength and what they most desired to do. As a result, every group member contributed to the thesis, and everyone worked together to complete the report on time. 

\subsection{Goals}
P1: \textit{Collaborate effectively with team members to ensure the timely completion of tasks} was achieved. As stated previously, the group incorporated a Kanban board into the Scrum framework allowing for the assignment and tracking of tasks. Additionally, the group utilized various communication platforms, such as Discord and Teams, to ensure effective communication among the group members. Using communication tools already integrated into each member's daily workflow was essential, and these two platforms proved to be the most effective for the team's needs. 
\\~\\
P2: \textit{Successfully integrating security tools (e.g., SAST, DAST, SCA) into the SDLC pipeline} was achieved. The group found tools that could be integrated into the pipeline between GitHub and AWS to secure the application code being sent through.
\\~\\
P3: \textit{Implement an automated pipeline using Terraform to build, test, and deploy applications}, was achieved. The group created Terraform code that automated the pipeline from the build to the deployment. 
\\~\\
R1: \textit{Develop a secure and automated pipeline for the SDLC process using Terraform}, is partially achieved. The group developed an automated pipeline using Terraform code and implemented restricted access management to ensure pipeline security. However, it would be presumptuous to claim that the pipeline is 100\% secure, as the group did not have sufficient time to implement signed artifacts. This would have ensured that the code pushed to the GitHub repository was the same as the code sent through the pipeline, thus guaranteeing code integrity. Consequently, while the pipeline can be deemed secure, the group cannot be entirely specific that the code sent through is secure. Hence, the goal of achieving a complete, secure, and automated pipeline remains partially fulfilled. 
\\~\\
R2: \textit{Produce a report summarizing the project results and recommendations for improving the SDLC pipeline security}, was accomplished to a certain degree. The resulting report provides numerous essential and effective practices that can be implemented to improve security in the \acrshort{sdlc}, and the goal was achieved within the requirements given. 

\section{Further Work}
For further work, the thesis could be strengthened by performing a broader analysis of various security tools that perform SAST, DAST, and SCA scans -  where the selected tools are based on these analyses. During these analyses, an assessment can also be made of which requirements must be met for a tool to be selected. 
\\~\\
Including earlier phases in the thesis would have been beneficial to acquire a more thorough grasp of the entire \acrlong{sdlc} and adhere to the shift-left methodology, which emphasizes early testing to find vulnerabilities earlier. In addition, this would have illustrated the shift-left method of undertaking to test as early as possible in the process.  
\\~\\
Despite successfully automating most of the pipeline, the group encountered challenges in automating Secret Scanning in GitHub. The lack of documentation on enabling it through \acrshort{cli} or Terraform code hindered the progress. Consequently, the group prioritized other tasks and allocated more time to this aspect in future work. The ultimate goal remains to achieve a fully automated pipeline. 

\newpage
\section{Conclusion}
The group is pleased to state that they have completed their thesis project, meeting the requirements set by their stakeholder while staying within the project's scope. The input from the stakeholders was critical in determining the group's objectives and requirements, resulting in a successful outcome that met the expectations of everyone involved.
\\~\\
After careful consideration, the group found that integrating Dependabot and CodeQL on GitHub was the most viable option compared to using \acrshort{sca} and \acrshort{sast} tools. The implementation process was smooth, and the group encountered no significant issues throughout the integration. In addition, despite GitHub's lack of a \acrshort{dast} scanning option, the group could utilize OWASP ZAP, which worked flawlessly with AWS and was easy to set up.
\\~\\
The group is confident that the stakeholder will significantly benefit from their thesis, and they highly recommend that the stakeholder consider implementing some of their recommendations in their daily work. The group is proud of their accomplishments and hopes that their work will significantly help the stakeholder.
