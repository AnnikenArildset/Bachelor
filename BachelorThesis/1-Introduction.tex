\chapter{Introduction}

\section{Background} %Hvorfor vi fikk oppgaven 
\acrlong{nbim}, from now on referred to as \acrshort{nbim}, is a division within the central bank responsible for overseeing the Government Pension Fund of Norway, which has a worth of 13,000 billion Norwegian kroner \cite{nbimwebsite}. Due to its large value, the fund is a major target for potential malicious actors. It faces an average of three severe cyber attacks per day, totaling around 100,000 attacks each year. Out of these, more than 1,000 are considered significant threats \cite{nbimattacks}. Therefore, it is crucial for \acrshort{nbim}, as well as other organizations, to ensure the security of their systems and applications before deploying them into their cloud services. 

\acrlong{sdlc} (\acrshort{sdlc}), describes how software applications are built - from planning through implementation and running in production. It also includes ensuring security at the different stages of software development. In order to accommodate frequent deployments to production, it is important to automate the security testing by building it into the deployment pipeline. The security testing can further benefit from shift-left, where testing is done as early as possible in the pipeline. Given that source code can be accessed by anyone, it is important to consider potential vulnerabilities during the development process. Implementing a strong and secure software development life cycle is essential to prevent attacks from hackers and other malicious actors on your application 
\cite{sdlc}. Securing the \acrshort{sdlc} is a large and actively developed area with a lot of interest from the industry. Demonstrating the integration and practical application of various tools and methods can be beneficial for both \acrshort{nbim} and other organizations.

\subsection{Problema Area}

\section{Scope}
\section{Scope Limitation}
In the thesis, all phases of the \acrlong{sdlc} will be introduced. There will be an introduction of what these phases are and why it is important to secure each phase. However, considering the scope, only the later four phases will be in focus in the thesis. 

\section{Target Group}
The thesis has multiple target groups. Our primary target group is NBIM, our stakeholder for this thesis, for whom we will produce a comprehensive report on the task assigned to us. This report has the potential to benefit other organizations, and therefore, another target group is any organization that utilizes the SDLC and aims to improve its security.

\section{The Group’s Academic Background}
The project group is currently in the third and final year of a Bachelor's degree program in Digital Infrastructure and Cyber Security at NTNU, campus Gjøvik. Throughout the studies, the group have covered a diverse range of subjects such as risk management, ethical hacking, cybersecurity, and teamwork. These subjects have equipped the group with relevant knowledge for their thesis work.

\subsection{What was needed to learn}%Fikse tittel 
For this thesis the group needed to learn about several critical aspects of software development, which included  \acrshort{sdlc}, \acrlong{sast}, \acrlong{dast} and \acrlong{sca}. Understanding the \acrshort{sdlc} was essential for the group, since it was required to understand as a part of the scope. 

To be able to test the code that was sent through the pipeline the group needed to understand \acrshort{sast}, \acrshort{dast} and \acrshort{sca}, since these are different testing tools that help developers secure their software applications. By testing the code with these tools the group were able to identify potential vulnerabilities within the code. For these tests, the group familiarized themselves with in-built tools in Github that ran such tests. 

In addition, the group also needed to learn Terraform, which was used to create pipelines in Terraform and instead of working directly in AWSs \gls{GUI}. The reason why the group decided on this was because it wasn't too complicated learning how to use Terraform to build a pipeline. 



\section{Framework}

\section{Methodology} %Hvordan vi har jobbet
Ser jentene har skrevet at de fordelte det over fire faser, der første fase gikk til innsamling av data osv osv. 
\subsection{Research methods}
\subsubsection{Interviews/meetings}
\subsection{Literature study}

\section{Thesis structure}
\section{Chapters}
The first chapter is an introduction to the thesis.

The second chapter, called Theory, contains theory that we consider important to understand the rest of the thesis.



