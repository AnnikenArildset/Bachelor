\newpage
\thispagestyle{empty}
\mbox{}
\chapter{Introduction}

\section{Background} %Hvorfor vi fikk oppgaven 
\acrlong{nbim}, from now on referred to as \acrshort{nbim}, is a division within the central bank responsible for overseeing the Government Pension Fund of Norway, which has a worth of 14,000 billion Norwegian kroner \cite{nbimwebsite}. Due to its significant value, the fund is a major target for potential malicious actors. It faces an average of three severe cyber attacks daily, totalling around 100,000 attacks each year. Out of these, more than 1,000 are considered significant threats \cite{nbimattacks}. Therefore, it is crucial for \acrshort{nbim}, as well as other organizations, to ensure the security of their systems and applications before deploying them into their cloud services. 
\\~\\
\acrlong{sdlc} (\acrshort{sdlc}) describes how software applications are built - from planning through implementation and running in production. It also includes ensuring security at the different stages of software development. In order to accommodate frequent deployments to production, it is essential to automate the security testing by building it into the deployment pipeline. Security testing can further benefit from shift-left, where testing is done as early as possible in the pipeline. Implementing a strong and secure software development life cycle is essential to prevent attacks from hackers and other malicious actors on the application. Securing the \acrshort{sdlc} is a large and actively developed area with much industry interest. Demonstrating the integration and practical application of various tools and methods can benefit both \acrshort{nbim} and other organizations.

\subsection{Problem area}
The technology industry is in constant development. With this development comes a rapidly expanding threat landscape. How developers approach IT and security must accompany this rapid development to secure systems and applications from malicious actors. Securing the \acrlong{sdlc} has become a common topic in the tech industry, and many resources are available to help organizations implement best practices and adequate security measures. However, finding the appropriate resources can take time, and automating the complete distribution process and automated testing using multiple tools can be time-consuming. Therefore, NBIM is looking for proof of concepts for building a secure pipeline using best practices and implementing multiple security tools to scan for security misconfigurations and vulnerabilities at crucial pipeline stages. 

 
\section{Scope limitations}
The thesis covers all phases of the \acrlong{sdlc}, and gives an overview of each phase and the importance of securing them. The phases consist of planning, implementation, testing, deployment, and maintenance. However, due to the scope, only the last four phases will be the main priority for testing purposes and building a secure pipeline from GitHub to \acrshort{aws}. The group has also decided not to focus on \gls{shift-left} testing within the life cycle, as the focus is on the later phases. 
\\~\\
As a part of the thesis, the group utilized \acrshort{aws} as it was required to use. However, considering the vastness of the \acrshort{aws} platform, the group chose not to explore the tools available within \acrshort{aws} extensively. Instead, the group opted to use the tools necessary at that time and were relatively simple to implement rather than focusing too much on the tools one could use to build the pipeline.
\newpage
\section{Target group}
The thesis has multiple target groups. The primary target group is \acrshort{nbim}, the stakeholder for this thesis, for whom the group will produce a comprehensive report on the task assigned to them. However, this report has the potential to benefit other organizations, and therefore, another target group is any organization that utilizes the \acrshort{sdlc} and aims to improve its security.

\section{Goals}
\subsection{Performance goals}
\begin{itemize}
    \item[-] P1: Collaborate effectively with team members to ensure the timely completion of tasks. 
    
    \item[-] P2: Successfully integrating security tools (e.g., \acrshort{sast}, \acrshort{dast}, \acrshort{sca}) into the \acrshort{sdlc} pipeline. 
    
    \item[-] P3: Implement an automated pipeline using Terraform to build, test, and deploy applications.
\end{itemize}

\subsection{Result goals}
\begin{itemize}
    
    \item[-] R1: Develop a secure and automated pipeline for the \acrshort{sdlc} process using Terraform. 
    
    \item[-] R2: Produce a report summarizing the project results and recommendations for improving the \acrshort{sdlc} pipeline security.
    
   % \item R3: The stakeholder start using the setup the group has created. 
   % \item R4: Give the stakeholder an overview of tools that can be used for security checks, making the process more automated and efficient. 
\end{itemize}


\section{The group's academic background}
The group is in the third and final year of a bachelor's degree program in Digital Infrastructure and Cyber Security at NTNU Gjøvik. Throughout the studies, the group has covered various courses such as risk management, ethical hacking, cyber security and teamwork. These subjects have equipped the group with relevant knowledge for their thesis work.

\subsection{Knowledge that had to be acquired}
\label{section: Knowledge that had to be acquired}
The group had to acquire various new aspects of software development for the thesis, as it was not covered extensively in their studies. These aspects include topics like \acrshort{sdlc}, \acrlong{sast}, \acrlong{dast} and \acrlong{sca}, and more. Due to the inclusion of the practical parts in the thesis, like testing insecure code, the group had to learn about building a pipeline and integrating various security tools to ensure a secure development. As a significant part of the main scope focused on tools integrated into GitHub and \acrlong{aws}, the group had to acquaint themselves with these tools and the various features of GitHub. Despite the group's experience with GitHub from previous courses, numerous features were still unfamiliar. In contrast, \acrshort{aws} was unfamiliar, and the group had no prior knowledge. However, the group had used similar cloud services like Microsoft Azure in previous courses.
\\~\\
In addition, the group had to familiarize themselves with Terraform and use it to establish the automated pipeline. Using Terraform instead of working solely in the \gls{GUI} was based on several reasons to use \gls{iac} over manual configuration. For instance, automation could enhance efficiency and decrease errors in the development process.

\subsubsection{Why this task was chosen}
The group chose this task because of the shared interest in the \acrlong{sdlc}. \acrshort{sdlc} is complex and contains multiple stages, and ensuring its security can be considered important in today's digital landscape. By looking at tools integrated in Github and \acrshort{aws}, the group aimed to understand better how to use these tools to identify and mitigate potential security risks at the different stages of the \acrshort{sdlc}.

\newpage
\section{Framework}

\subsection{Timeframe}
The task completion deadline runs from the 11th of January, 2023, to the 22nd of May, 2023. It was agreed early on that the first draft would be submitted to the supervisor 3rd of April to allow the supervisor enough time to read through the thesis and give feedback. Then, the group decided to submit the final draft to the supervisor on the 1st of May, which would give a month between the first draft and the final draft to work.    


\subsection{Other}
The stakeholder requested a comprehensive report that did not specifically focus on their systems, as they were unable to provide access to their systems or development environment for the group. Therefore, the group had a significant amount of freedom in determining the scope and specific limitations of the thesis. 

\section{Methodology}
The group adopted a \acrshort{devsecops} approach as a working method during the project, which includes incorporating security into the DevOps process at all stages of the \acrshort{sdlc}. Despite focusing solely on the last four phases of the \acrshort{sdlc}, the team maintained a DevSecOps mindset, viewing security as an essential component of the entire process rather than a separate phase. 


\section{Research methods}
The group used different methods to gain knowledge that was needed to fulfill the requirements that were given. Below are some of the research methods used to gain knowledge about the topic. 

\subsection{Interviews/meetings}
At the beginning of the project, the group conducted interviews with various lecturers at NTNU Gjøvik to collect information on relevant subjects. By getting insights from software security experts, the group acquired a fresh perspective on the topic, supplementing the knowledge gained through literature studies. Additionally, the group prepared a list of pre-determined questions to address any uncertainties about specific areas that remained unclear based on prior research.  

\subsection{Literature study}
The literature study was done by researching online for relevant information about the topics, which helped the group add suitable material for the theory chapter. In addition, the study provided the essential knowledge required to address secure \acrshort{sdlc} practices, which formed the basis for delivering appropriate recommendations.

\section{Software utilized for writing}
This section contains an overview of software utilized in the process of writing the thesis.
\begin{itemize}
    \item \textbf{ChatGPT:} Primarily used for rephrasing when uncertain how to formulate the message.
    \item \textbf{Grammarly Premium:} For correct grammar, consistency and precise wording.
    \item \textbf{Overleaf:} Cooperative LaTeX editor used for writing the thesis. 
\end{itemize}

\section{GitHub organization}
As part of the thesis, the group created a GitHub organization containing different repositories, including all the code utilized for creating an automated pipeline and various security tools and a backup of our thesis. 

\href{https://github.com/orgs/DCSG2900-Bachelor-thesis/repositories}{\textbf{https://github.com/orgs/DCSG2900-Bachelor-thesis/repositories}}

\section{Thesis structure}
When reading the thesis, some things can be helpful to note. First, there are clickable links that make the navigation to chapters, sections, acronyms, figures, tables, and sources go as seamlessly as possible. The language used throughout the thesis is English. The same goes for all meeting notes and other relevant appendixes. 


\subsection{Chapters}
\begin{itemize}
    \item \textbf{Chapter 1 - (Introduction)}: Contains an introduction to the thesis.
    \item \textbf{Chapter 2 - (Theory)}: Contains theory that the group considers important to have some knowledge about to understand the thesis as a whole.
    \item \textbf{Chapter 3 - (Pipeline security)}: Contains an overview of what to implement to secure data sent through the pipeline and what to include to secure the pipeline. 
    \item \textbf{Chapter 4 - (Analysis of security tools for the pipeline)}: Contains an overview of the different tools the group has implemented into their pipeline, with an analysis of these tools.
    \item \textbf{Chapter 5 - (End-to-end pipeline)}: Contains the steps done for pipeline automation and implementation of chosen security tools. 
    \item \textbf{Chapter 6 - (Discussion)}: Contains an in-depth discussion of choices made throughout the thesis. 
    \item \textbf{Chapter 7 - (Conclusion)}: Contains an overview of the work process and a conclusion to the group's findings. 

\end{itemize}






