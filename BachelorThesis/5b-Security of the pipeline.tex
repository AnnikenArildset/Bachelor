\section{Security of the pipeline}
Security of the pipeline in short terms are the security measures that are taken into consideration when securing the pipeline itself. This includes not only securing the pipeline istelf, but also the infrastrucutre, components, network that are used to process the code that goes through the pipeline. Securing of the pipeline ensures that the code is not tampered through during the process of going through the pipeline. 

\subsection{Signed Commits}

\subsection{Branch Protection}

\subsubsection{Require a pull request before merging}
Administrators of the repository can add rules to the repository which restricts pull requests to have a specific number of people approving the changes before merging to a protected branch. Administrators can allow users with written permissions to do the approving as well as users considered to be code owners. \cite{ProtectedBranches}

It is under this type of protection the "four eyes" principal is applied. Since this type of protection require that at least two people approve the merge, this includes the person itself doing the changes, this principal is a controling mechanism that increases the security measures. 
\subsubsection{Require status checks to pass before merging}
hei
\subsubsection{Require conversation resolution before merging}
hei
\subsubsection{Require signed commits}
hei
\subsubsection{Require linear history}
hei
\subsubsection{Require deployments to succeed before merging}
hei
\subsubsection{Lock branch}
hei
\subsubsection{Do not allow bypassing the above settings}
hei

\newpage



\subsection{Access Control}
\subsection{File Storage and Preservation}