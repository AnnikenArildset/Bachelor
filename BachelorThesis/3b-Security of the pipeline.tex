\section{Security of the pipeline}
\label{Security of the pipeline}
Security of the pipeline refers to the measures taken to protect not only the pipeline, but the underlying infrastructure, and network involved in processing the code that passes through it. By ensuring the security of the pipeline, it is restricted to performing only its intended functions and preventing unauthorized access to restricted resources.  

\subsection{Branch Protection}
Branch protection ensures that certain criteria are met before merging code. This feature allows users to create branch protection rules that enforce specific workflows for one or multiple branches, such as mandating an approving review or passing status checks for all pull requests merged into the protected branch. Access to this feature of GitHub is available for all users.\cite{branch}
\\
By enforcing branch protection, the chances of introducing errors and vulnerabilities into the secured branch are less likely. Branch protection also creates a clearer development process by providing a set of guidelines and requirements for making changes in the code. This can help ensure that all team members are on the same page and working towards the same goal. 
%\label{branchprotection}
%Branch protection is a feature of GitHub, that enforces different rules and requirements for specific branches in the repository. The purpose of branch protection is to maintain the security of the code, this is done by ensuring that all changes done to the branch have gone through the proper steps before being merged into the main branch. Figure \ref{fig: Pipeline with implemented branch protection rules} demonstrates that enabling of branch protection is done before code is pushed to GitHub, in order for the source code to enter a secure repository. Below are the different branch protection features that can be enabled in GitHub. \cite{ProtectedBranches}

%%Administrators of the repository can add rules to the repository which restrict pull requests to have a specific number of people approving the changes before merging to a protected branch. Administrators can allow users with written permissions to do the approving as well as users considered to be code owners. 
%\\~\\
%It is under this type of protection that the "Four Eyes Principle" is applied. Since this type of protection require that at least two people approve the merge, this includes the person itself doing the changes. This principle can be considered a controlling mechanism that improves the quality of the outcome, minimize risk errors and prevents malicious actions by a single individual. 

%\subsubsection{Require status checks before merging}
%Maintaining high code quality is important when multiple users collaborate within a shared repository. 
%By enabling \say{require status checks to pass before merging} feature, repository administrators can establish specific criteria that must be met before code is merged, such as requiring code approval from at least one team member.

%\subsubsection{Require conversation resolution before merging}
%When working together on the same repository it is important to have clear communication and collaboration. A way to secure this is to enable \say{require conversation resolution before merging}. This allows all discussions regarding for example issues or pull requests that need to be properly resolved before any merging happens. 

%\subsubsection{Require signed commits}
%Enabling require \say{signed commits} can be considered a security measure that ensures that changes in the code have not been tampered with. 
%To be able to have secured signed commits, all commits pushed to the repository must be signed with a \acrlong{gpg} (\acrshort{gpg}) key or an \acrshort{ssh} key. 


%\subsubsection{Require deployments to succeed before merging}
%\say{Require deployments to succeed before merging}enables users to enforce passing of various required checks, such as pre-merge checks or automated tests, before allowing a pull request to be merged into the main branch.

%\subsubsection{Lock branch}
%Lock branch as the name implies allows the users to lock a branch in a repository, which will prevent changes from being made to the branch. This can be useful if there are situations where the branch needs to be protected from unauthorized changes or to be deleted. 

%\subsubsection{Do not allow bypassing the above settings}
%This feature stops users from bypassing required checks and restrictions in a repository. For example, if an administrator sets a rule that all pull requests must pass reviews and checks before merging, the feature prompts users to comply before making changes to the branch.


\vspace{2mm}
\begin{figure}[H]
    \centering
    \includegraphics[width=0.8\columnwidth]{Images/pipeline6.png}
    \caption{Pipeline with implemented branch protection rules}
    \label{fig: Pipeline with implemented branch protection rules}
\end{figure}

\subsection{Access Control}
Access control involves implementing measures to regulate the individuals who are allowed to access certain resources like GitHub, as well as determining the appropriate level of access for each individual. When applying these measures, it is recommended to follow the \say{least privilege} principle. According to NIST \cite{leastprivilege}, this is \textit{"the principle that a security architecture should be designed so that each entity is granted the minimum system resources and authorizations that the entity needs to perform its function"}. In Github, access is controlled by permission - which is the capability to execute a particular task. Employees can also be assigned to roles. A role is a set of permissions that can granted to an employee or a team. \cite{accesscontroll}
\\~\\
Secure authentication is a crucial aspect of maintaining system security. It involves implementing measures to ensure that only authorized users can access a system, and that their access is limited to the specific resources that they need to perform their job duties.
\\~\\
One of the ways that organizations can enhance their authentication process is through the use of conditional access. This involves setting rules and conditions that determine when, where, and how users can access a system. For example, an organization might require multi-factor authentication for all users who are accessing the system from outside the corporate network, or they might limit access to certain applications or data based on the user's role or location.
\\~\\
By using conditional access, organizations can more effectively control access to their systems, reduce the risk of unauthorized access or data breaches, and ensure compliance with regulatory requirements. It is an important part of a comprehensive security strategy, and should be implemented across all system components to provide maximum protection. Access control should be done before the code enters any software, in this case GitHub and \acrshort{aws}, to ensure the regulation of access at any time, as shown in Figure \ref{fig: Pipeline with implemented access control}.

\vspace{2mm}
\begin{figure}[H]
    \centering
    \includegraphics[width=0.8\columnwidth]{Images/pipeline7.png}
    \caption{Pipeline with implemented access control}
    \label{fig: Pipeline with implemented access control}
\end{figure}
 
\subsection{File Storage and Preservation}
To be able to maintain proper security of the pipeline, file storage, and preservation can be considered a large part of this. File storage and preservation help secure the pipeline over time, by doing regular backups of the data related to the pipeline. By maintaining redundant copies of critical data, the risk of data loss due to attacks or other security breaches can be minimized. 
\\~\\
Some other important aspect of file storage and preservation is access control, which should be restricted so that individuals only have the necessary access. As a result, this can minimize the chances of data tampering or data theft.

\section{Security in maintenance}
Once the deployment is complete, the application is then transferred to the cloud environment in \acrlong{aws}. It is essential at this stage to keep the security up to date, to ensure that the application and data are protected. AWS offers a range of best practices organizations can follow to decrease risks associated with cloud computing and ensure that the  \acrshort{aws} environment is secure.
\\~\\
After the deployment is successful, it's essential to ensure that the infrastructure is secure and optimized for performance. Regular maintenance and updates are important to solve security issues and add new features to the system. It's also important to perform regular backups and disaster recovery testing to ensure that the data and applications are protected. 
Staying up-to-date with maintenance and testing can minimize downtime and improve system reliability. 
\\~\\
When the application is in the cloud environment, it is important to continuously monitor it. This is to identify any issues or potential threats. This includes monitoring for errors, performance issues, security vulnerabilities, and more. \acrshort{aws} offers specialized tools for this type of monitoring, like AWS Cloudtrail\footnote{Available at: \url{https://aws.amazon.com/cloudtrail/}} and AWS X-ray\footnote{Available at: \url{https://aws.amazon.com/xray/}}. 
\\~\\
Various security tests, such as security scans and penetration testing, should be performed during the development phases to identify potential vulnerabilities and address them before deployment. However, it is also essential to continue testing after the deployment to ensure that any new vulnerabilities are identified and addressed as soon as possible. Regular security scans and penetration testing can significantly reduce the risk of exploitation. By conducting these tests regularly, the organization can keep track of any possible security problems and take proactive measures to mitigate them.
\\~\\
As a protective measure, the organization should set up a web application firewall like AWS WAF\footnote{Available at \url{https://aws.amazon.com/waf/}}, to prevent malicious application attacks such as \gls{SQL-injection}, \gls{Cross-site scripting} and other attacks. AWS's WAF service offers a managed set of protective rules, allowing customized rules and access control lists based on the company's needs and risk models. This makes it possible to provide web application security with more customization and specificity. Securing an application against \gls{ddos} is an important additional step. AWS Shield\footnote{Available at \url{https://aws.amazon.com/shield/}} is a service that an organization can utilize to achieve this. \cite{awsafterdep}
\\~\\
Security is an important aspect of the SDLC's maintenance phase, which begins after the application has been deployed. It is critical to keep up with regular maintenance and testing during this phase, as well as to take proactive measures to mitigate any issues that may arise. Organizations can identify and address potential security vulnerabilities in this manner before they become major issues. This is especially important when deploying applications to the cloud because the security environment can be complex and ever-changing. Best practices such as regular security audits, vulnerability scans, and the implementation of security patches can help to ensure that the application remains secure and protected against potential threats. Monitoring for unusual or suspicious activity can also aid in the detection and prevention of security breaches. Organizations can help to ensure the continued reliability and security of their AWS applications by prioritizing security throughout the maintenance phase.

\section{Finished pipeline}
Figure \ref{fig: Pipeline with all security measures implemented} illustrates a finished pipeline after all security measures are included. This also includes maintenance.  

\vspace{2mm}
\begin{figure}[H]
    \centering
    \includegraphics[width=0.8\columnwidth]{Images/pipeline9.png}
    \caption{Pipeline with all security measures implemented}
    \label{fig: Pipeline with all security measures implemented}
\end{figure}



