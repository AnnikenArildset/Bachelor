\newpage
\label{EnableCodeQL}
\begin{lstlisting}[language = bash, caption=Enable CodeQL]

#!/bin/bash

#This command authenticates the user.
gh auth login

#This command enables CodeQL in GitHub. 
gh api -X PATCH /repos/ceelinab/juice-shop/code-scanning/default-setup
-f state=configured

#This command allow the user to monitor the actions happening in the repository. 
gh api /repos/ceelinab/juice-shop/actions/runs/4869392483 
--jq '.status,.conclusion'

#Checks if it was successful
gh api /repos/ceelinab/juice-shop/code-scanning/default-setup

\end{lstlisting}
\newpage
\section{Comments to the script}
The script below is designed to work for the group, but will not necessary work for others that does not have access to the repository used in this case. 
However, the syntax is the same, so below will one find the syntax for the different commands. It is also important to mention that the "run-id" is the id that is received  from when you run the second command below. 

\begin{tcolorbox}
\begin{verbatim}
 gh auth login

gh api -X PATCH /repos/[org-name]/[repo-name]/code-scanning/
default-setup -f state=configured

gh api /repos/[org-name]/[repo-name]/actions/runs/run_id 
--jq '.status,.conclusion'

gh api /repos/[org-name]/[repo-name]/code-scanning/default-setup

 
\end{verbatim}
\end{tcolorbox}