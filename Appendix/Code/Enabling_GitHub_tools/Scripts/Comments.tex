\newpage
\section{Comments to the script}
The commands given above is designed to work for the group, but will not necessary work for others that does not have access to the repository used in this case. 
However, the syntax is the same, so below one will find the syntax for the different commands. It is also important to mention that the "run-id" is the id that is received  from when you run the second command below. 

\begin{tcolorbox}
\begin{verbatim}
#Authenticate
 gh auth login
#Enables CodeQL with basic setup
gh api -X PATCH /repos/[org-name]/[repo-name]/code-scanning/
default-setup -f state=configured

#Monitors the change
gh api /repos/[org-name]/[repo-name]/actions/runs/run_id 
--jq '.status,.conclusion'

#Checks if it was successful 
gh api /repos/[org-name]/[repo-name]/code-scanning/default-setup

___________________________________________
#Enable Dependabot
gh api -X PUT /repos/[org-name]/[repo-name]/vulnerability-alerts
 
\end{verbatim}
\end{tcolorbox}